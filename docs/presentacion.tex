\documentclass[aspectratio=169]{beamer}
\usepackage[utf8]{inputenc}
\usepackage[spanish]{babel}
\usepackage{amsmath}
\usepackage{amssymb}
\usepackage{graphicx}
\usepackage{booktabs}
\usepackage{tikz}
\usetikzlibrary{shapes,arrows,positioning}

% Tema moderno con colores cyan/azul oscuro
\usetheme{Madrid}
\usecolortheme{default}

% Definir colores personalizados
\definecolor{darkbg}{RGB}{10,14,39}
\definecolor{mediumbg}{RGB}{15,23,42}
\definecolor{accent}{RGB}{6,182,212}
\definecolor{accentlight}{RGB}{34,211,238}
\definecolor{textcolor}{RGB}{241,245,249}

% Aplicar colores personalizados
\setbeamercolor{palette primary}{bg=darkbg,fg=textcolor}
\setbeamercolor{palette secondary}{bg=mediumbg,fg=textcolor}
\setbeamercolor{palette tertiary}{bg=accent,fg=white}
\setbeamercolor{palette quaternary}{bg=accentlight,fg=darkbg}
\setbeamercolor{structure}{fg=accent}
\setbeamercolor{title}{fg=accent}
\setbeamercolor{frametitle}{fg=accent}
\setbeamercolor{normal text}{fg=textcolor,bg=darkbg}
\setbeamercolor{block title}{bg=accent,fg=white}
\setbeamercolor{block body}{bg=mediumbg,fg=textcolor}
\setbeamercolor{itemize item}{fg=accent}
\setbeamercolor{itemize subitem}{fg=accentlight}
\setbeamercolor{enumerate item}{fg=accent}

% Quitar símbolos de navegación
\setbeamertemplate{navigation symbols}{}

% Numeración de slides
\setbeamertemplate{footline}[frame number]

% Título
\title{Sistema de Minimización de Polarización}
\subtitle{Optimización mediante Programación Entera Mixta}
\author{Andrey Quiceño \and Iván \and Francesco \and Jonathan}
\institute{Universidad del Valle \\ Análisis de Algoritmos II}
\date{Diciembre 2025}

\begin{document}

% Slide 1: Portada
\begin{frame}
\titlepage
\end{frame}

% Slide 2: Problema y Motivación
\begin{frame}{El Problema de la Polarización}
\begin{block}{Contexto}
\begin{itemize}
    \item Población con $n$ personas distribuidas en $m$ opiniones
    \item Cada opinión tiene un valor $v_i \in [0,1]$
    \item Personas con diferentes niveles de resistencia al cambio
\end{itemize}
\end{block}

\begin{block}{Objetivo}
Minimizar la polarización final mediante movimientos estratégicos entre opiniones, sujeto a:
\begin{itemize}
    \item Presupuesto limitado ($c_t$)
    \item Número máximo de movimientos
    \item Costos variables según resistencia
\end{itemize}
\end{block}
\end{frame}

% Slide 3: Modelo Matemático
\begin{frame}{Modelo Matemático}
\begin{block}{Parámetros}
\begin{itemize}
    \item $n$: total de personas, $m$: opiniones
    \item $p_i$: personas en opinión $i$
    \item $s_{i,k}$: personas en opinión $i$ con resistencia $k$ ($k=1,2,3$)
    \item Factores: baja(1.0), media(1.5), alta(2.0)
\end{itemize}
\end{block}

\begin{block}{Variables de Decisión}
$$x_{k,i,j} \geq 0 \quad \forall k \in \{1,2,3\}, i,j \in \{1,\ldots,m\}$$
Número de personas con resistencia $k$ que se mueven de $i$ a $j$
\end{block}
\end{frame}

% Slide 4: Función Objetivo
\begin{frame}{Función Objetivo y Mediana}
\begin{block}{Polarización}
$$\text{Pol}(p',v) = \sum_{i=1}^{m} p'_i \cdot |v_i - \text{mediana}(p',v)|$$
Donde $p'_i$ es la distribución final de personas
\end{block}

\begin{block}{Cálculo de Mediana}
Enfoque acumulativo con restricciones:
\begin{itemize}
    \item $cum_1 = p'_1$
    \item $cum_i = cum_{i-1} + p'_i$ para $i \geq 2$
    \item Posición mediana: $median\_pos = \lfloor (n+1)/2 \rfloor$
    \item $median\_opinion$: primera opinión donde $cum_i \geq median\_pos$
    \item $median\_value = v_{median\_opinion}$
\end{itemize}
\end{block}
\end{frame}

% Slide 5: Restricciones del Modelo
\begin{frame}{Restricciones del Modelo}
\begin{enumerate}
    \item \textbf{Capacidad por resistencia:}
    $$\sum_{j=1}^{m} x_{k,i,j} \leq s_{i,k} \quad \forall k, i$$
    
    \item \textbf{No auto-movimientos:}
    $$x_{k,i,i} = 0 \quad \forall k, i$$
    
    \item \textbf{Conservación de población:}
    $$p'_i = p_i + \sum_{k,j} (x_{k,j,i} - x_{k,i,j})$$
    
    \item \textbf{Costo total:}
    $$\sum_{k,i,j} r_k \cdot x_{k,i,j} \leq c_t$$
    
    \item \textbf{Movimientos máximos:}
    $$\sum_{k,i,j} x_{k,i,j} \leq \text{maxMovs}$$
\end{enumerate}
\end{frame}

% Slide 6: Algoritmo Branch and Bound
\begin{frame}{Branch and Bound con Gecode}
\begin{columns}
\begin{column}{0.5\textwidth}
\textbf{Estrategia de Búsqueda}
\begin{itemize}
    \item \textbf{Variable selection}: first\_fail
    \item \textbf{Value choice}: indomain\_min
    \item \textbf{Solver}: Gecode
\end{itemize}
\end{column}
\begin{column}{0.5\textwidth}
\textbf{Proceso}
\begin{enumerate}
    \item Exploración del árbol
    \item Branching en $x_{k,i,j}$
    \item Podas por restricciones
    \item Actualización de cota
    \item Convergencia a óptimo
\end{enumerate}
\end{column}
\end{columns}

\vspace{0.5cm}
\begin{block}{Resultado}
Solución óptima garantizada con polarización mínima
\end{block}
\end{frame}

% Slide 7: Implementación
\begin{frame}{Arquitectura del Sistema}
\begin{columns}
\begin{column}{0.5\textwidth}
\textbf{Componentes}
\begin{itemize}
    \item \texttt{model/Proyecto.mzn}: Modelo MiniZinc
    \item \texttt{input\_output/}: Parser I/O
    \item \texttt{gui.py}: Interfaz gráfica
    \item \texttt{scripts/}: Testing y build
\end{itemize}
\end{column}
\begin{column}{0.5\textwidth}
\textbf{Tecnologías}
\begin{itemize}
    \item MiniZinc 2.6+
    \item Python 3.8+
    \item Tkinter (GUI)
    \item PyInstaller (.exe)
\end{itemize}
\end{column}
\end{columns}

\vspace{0.5cm}
\begin{block}{Flujo}
Archivo .txt $\rightarrow$ Parser $\rightarrow$ .dzn $\rightarrow$ MiniZinc $\rightarrow$ Resultado $\rightarrow$ GUI
\end{block}
\end{frame}

% Slide 8: Resultados y Pruebas
\begin{frame}{Validación y Resultados}
\begin{block}{Batería de Pruebas}
\begin{itemize}
    \item 35 casos de prueba
    \item Rangos: $n \in [10, 100]$, $m \in [3, 5]$
    \item Validación automática contra resultados esperados
    \item Tolerancia: 0.001
\end{itemize}
\end{block}

\begin{block}{Desempeño}
\begin{itemize}
    \item Casos pequeños ($n \leq 20$): $< 1$ segundo
    \item Casos medianos ($20 < n \leq 50$): 1-10 segundos
    \item Casos grandes ($n > 50$): 10-60 segundos
    \item Timeout configurado: 5 minutos
\end{itemize}
\end{block}
\end{frame}

% Slide 9: Demo de la Interfaz
\begin{frame}{Interfaz Gráfica}
\begin{block}{Características}
\begin{itemize}
    \item Diseño moderno con tema cyan/azul oscuro
    \item Carga de archivos .txt intuitiva
    \item Visualización de parámetros
    \item Ejecución asíncrona (no bloquea UI)
    \item Resultados con formato profesional
    \item Exportación a archivos
\end{itemize}
\end{block}

\begin{block}{Uso}
\begin{enumerate}
    \item Seleccionar archivo de entrada
    \item Cargar y visualizar datos
    \item Ejecutar optimización
    \item Revisar polarización y movimientos
    \item Guardar resultado
\end{enumerate}
\end{block}
\end{frame}

% Slide 10: Conclusiones
\begin{frame}{Conclusiones y Trabajo Futuro}
\begin{block}{Logros}
\begin{itemize}
    \item[$\checkmark$] Modelo completo y funcional en MiniZinc
    \item[$\checkmark$] Implementación robusta con manejo de errores
    \item[$\checkmark$] Interfaz profesional y amigable
    \item[$\checkmark$] 35 pruebas validadas exitosamente
    \item[$\checkmark$] Ejecutable portable para Windows
\end{itemize}
\end{block}

\begin{block}{Extensiones Posibles}
\begin{itemize}
    \item Análisis de sensibilidad de parámetros
    \item Optimización multiobjetivo
    \item Visualización gráfica de resultados
    \item Modelos de predicción temporal
\end{itemize}
\end{block}

\vspace{0.3cm}
\centering
\large \textbf{¿Preguntas?}
\end{frame}

\end{document}
